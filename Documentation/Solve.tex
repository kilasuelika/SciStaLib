\chapter{Optimization and Equation System Solving(Solve)}
\section{Usage}

\subsection{Optimization}
\subsubsection{Alogorithms}
There are r types of algorithms: unbounded, bounded, linear constrained and nonlinear contrained. Each algorithm is implemented in an object and we pass problem functions to these objects. We use a problem function to return all related information including function value, gradients, nonlinear constraint values etc. To return these information, we use a tuple as return type.

A list of algorithms:

\begin{enumerate}
	\item Unbounded. Heuristic: \cd{GA},  \cd{PSO}.
	\item Bounded: 
	\item Linear Constrained: 
\end{enumerate}

\subsubsection{Variable Types}
\begin{enumerate}
	\item \textbf{x, y, gradient} are forced to be \cd{VectorXd} type and \textbf{hessian} are forced to be \cd{MatrixXd} type. If \textbf{x,y} of objective function or nonlinear constraints are scalars, we still use a length one \cd{VectorXd} to store it. One advantage of this is that we don't have to deal with number of variables. Note that \cd{VectorXd} is a \textbf{column} vector.
	
	If \textbf{y} is a vector of length more than one. One can set \cd{SolveOption.type} to value either of "least square" or "norm". For "norm", we optimize norm of \textbf{y}.
	
	\item One can pass external data by pointer using template.
\end{enumerate}

\subsubsection{GA  Examples}
\begin{lstlisting}
//Objective function.
double fun1(VectorXd x){
	return pow(x[0],2)+pow(x[1],2);
};

GA ga_optimizer("min", 2);

//Following are some  non-necessary settings. They have default values.
ga_optimizer.set_population();



VectorXd x0;
x0<<10,10;
auto flag=ga_optimizer.solve(fun1, x0);
cout<<x0<<endl;


\end{lstlisting}


\subsubsection{Write An Optimization Problem}
\begin{lstlisting}
class prob: ProblemBase<VectorXd>{
	
}
\end{lstlisting}

\subsubsection{Constrained Optimization}


\subsection{Nonlinear System}
\begin{lstlisting}

\end{lstlisting}

\section{Common Objects}
\subsection{SolveOption}

\subsection{SolveResult}

\section{Solve optimization problems}
\subsection{Choose A Solver}
If a solver requires gradient and hessian, however they are not provided, then default difference approximation will be used.

\begin{itemize}
	\item \textbf{Special}: \cd{QPSolver}(Constrained QP)
	\item \textbf{Unconstrained}:
	\item \textbf{Linear inequality constrained}:\cd{LCOBYQASolver}(doesn't use gradient and hessian).
	\item \texbt{Nonlinear constrained}:\cd{LSSQPSolver}(Large scale SQP).
	\item \textbf{Nonlinear least square}:
	\item \textbf{Evolutionary}:\cd{DESolver}.
	\item \textbf{Heuristic Method}: \cd{PSOSolver}.
\end{itemize}
\subsection{SolverBase}

\subsection{QPSolve}
\cd{QPSolve} is a basic component of many other solvers. It's not a derived class of \cd{SolverBase}. It solve a problem:
\begin{align}
\min_x & f(x)=\frac{1}{2}x^TGx+x^Tc \\
subject & Ax \leq b
\end{align}

Usage:

\begin{lstlisting}

\end{lstlisting}

\section{EQSystem - Solve nonlinear system of equations}
