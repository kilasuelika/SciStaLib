\chapter{Utility}

\section{EigenHelper}
\subsection{slice\_by\_set - indexing a matrix by integer set}
Giving a matrix \cd{mat} and a integer set \cd{is}, dimension indicator \cd{dim}. Return a matrix so that each column or row comes from \cd{mat} indicated by \cd{is}.

\cd{dim=0} for select rows. \cd{dim=1} for select columns.

\noindent Definition:
\begin{lstlisting}
template<typename SV, typename IDXT>
SV slice_by_set(SV mat, IDXT is, int dim=0);
\end{lstlisting}

\cd{IDXT} is the type of set. It can be \cd{set<int>} or \cd{vector<int>}. It's very useful when you want to manipulate a subset of rows or columns.

\noindent Example:
\begin{lstlisting}
set<int> s;
s.emplace(1);
s.emplace(2);
s.emplace(0);

MatrixXd m(5, 4);
m.setRandom();

cout << slice_by_set(m,s) << endl;
cout << slice_by_set(m, vector<int>{ 0,1,2 }) << endl;
\end{lstlisting}

\section{FunctionCollection}
\subsection{print\_stl - print values in a STL container}

\noindent Definition:
\begin{lstlisting}
template<typename T>
void print_stl(T& v);
\end{lstlisting}

\subsection{which - get indices of a value in a container}
\noindent Definition:
\begin{lstlisting}
template<typename V, typename S>
vector<int> which(const V& v, const S& val);
\end{lstlisting}

\subsection{set2vec - convert a set to vector}
\noindent Definition:
\begin{lstlisting}
template<typename T>
vector<T> set2vec(const set<T>& s);
\end{lstlisting}